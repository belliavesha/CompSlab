\documentclass[iop, usenatbib]{emulateapj}

%\documentclass{article}
\usepackage{amssymb}
\usepackage{epsfig}
\usepackage{color}
\usepackage{bm}
\usepackage{mathtools}
\usepackage{graphicx}
\usepackage{gensymb}
\usepackage{lipsum}
\usepackage{hyperref}
%\usepackage[showframe]{geometry}% http://ctan.org/pkg/geometry
%\usepackage{multicol}% http://ctan.org/pkg/multicols

\newcommand{\be}{\begin{equation}}
\newcommand{\ee}{\end{equation}}

%\slugcomment{ }
%\shorttitle{Polarization from neutron stars}
%\shortauthors{Me}

%\voffset=-1cm

\begin{document}
\title{Polarized radiation from electron slab atmosphere}



\author{ V. Loktev }

%\institute{ Me Me Big Boy}

\date{Received XXX / Accepted XXX}



\begin{abstract}
\lipsum[1]
\end{abstract}


\keywords{methods: numerical --- transfer:radiative  --- stars: neutron --- me me: big boy}

%\titlerunning{Meme BB}

\maketitle

\section{Compton scattering}\label{Compton}
\subsection{Setup}
The simple neutron star atmosphere model is considered in which the atmosphere is a plane parallel electron consisting slab lying on an optically thick source of thermalized  black body radiation.
The atmosphere slab has vertical hight $H$ which is equivalent to optical depth of thermalization layer $\tau_T=\sigma_Tn_eH$    
The electron gas in the atmosphere is considered to be isotropic and isothermal with constant dimensionless electron temperature $\Theta$ for any depth under surface from the bottom of the slab $\tau=0$ up to the surface $\tau=\tau_T$.
In case of neutron star the temperature $\Theta$ is corresponding to electron energies about $50 keV$ which is equivalent to $\Theta\approx 0.1$ in units of electron rest energy $m_ec^2$. 
Since the electron velocities are relativistic, they are characterized by Lorenz factor $\gamma$
The momentum distribution of the electrons is considered to be relativistic Maxwellian distribution characterized by reversed dimensionless temperature $Y = 1/ \Theta$ and given by 
\be 
\label{eq:Maxwell}
f_M(\gamma)= \frac{Y e^{-\gamma Y}}{4\pi K_2(Y)}
\ee
where $K_2$ is the second modified Bessel function.  
Radiation intensity and polarization degree can be described by Stokes vector $\bm{I}(\tau,x,\mu)$. $\mu$ is the cosine of an angle between vertical axis of the slab and direction of emergent photons propagation.  
The Stokes vector is usually contains 4 components, but due to angular symmetry of the star and absence of circular polarization source the last two of them will equal to zero.
Thus one can put $\bm{I}(\tau,x,\mu)=\bm{I}(z,\nu,\mu) = (I,Q)^T$.
The photon distribution on the bottom of the slab is considered to be Planckian with characteristic dimensionless temperature $T$.  In case of neutron star atmosphere this temperature is taken corresponding to photon energies around $1keV$ or $x\approx 0.002$ in  units of electron rest energy.
Thus the initial intensity on the bottom of the slab according to the Planck's law is
\be
I_{in}(\tau=0,x)=\frac{2m_e^4c^6}{h^3}\frac{x^3}{e^{x/T}-1}
\ee   
\section{Radiative transfer}
In order to describe the propagation of polarized radiation through a plane parallel electron slab we use the radiative transfer equation, which in our  simple case, when only Compton scattering is being considered while pair production and all sorts of emission are neglected, in terms of dimensionless quantities such as $$
 I'=I \frac{H \sigma_T}{m_e c^3}=I \frac{\tau_T}{n_e m_e c^3} \qquad S'=S \frac{H \sigma_T}{m_e c^3}$$$$
\sigma'_{CS}=\sigma_{CS}/\sigma_T \qquad d\tau= \sigma_T n_e dz $$
takes its dimensionless form 
\be
\label{eq:transfer}
\mu \frac{d I (\tau,x,\mu)}{d\tau} = -  \sigma_{CS}(x)I(\tau,x,\mu) + S(z,x,\mu) 
\ee
where $S$ is the source function, which can be expressed in terms of azimuth averaged Compton redistribution function $\bm{R}$, the 
computation of which is discussed in section \ref{redistr} 
\be
\label{eq:Source}
S(\tau,x,\mu)= x^2 \int_0^\infty \frac{dx_1}{x_1^2} \int_{-1}^1 d\mu_1 
\bm{R}(x,\mu,x_1,\mu_1)\bm{I}(\tau,x_1,\mu_1)
\ee
The last equation can be simply written in form of integral operator
\be
\label{eq:Rop}
\bm{S}=\hat{\bm{R}}\bm{I}
\ee
 The integrodifferential equation \eqref{eq:transfer} can be solved by expanding the solution $\bm{I}$ in scattering orders
 \be
 \bm{I}=\sum_{k=0}^\infty \bm{I}_k
\ee
 where $\bm{I}_k$ is the Stokes vector describing the photons  that have undergone $k$ scatterings. Each of them can also be written in form of integral operator  of $k$-th source function $\bm{S}_k$
\be
\label{eq:Sop}
 \bm{I}_k=\hat{\bm{\Sigma}}\bm{S}_k
\ee
which is
 \be
 \hat{\bm{\Sigma}}\bm{S}=\int_{\tau_TH(\mu)}^\tau \frac{d\tau'}\mu \bm{S}(\tau',x,\mu) e^{\sigma(x)\frac{\tau'-\tau}\mu}
\ee 
where $H(\mu)$ is Heaviside step function. The initial Stokes vector $\bm{I}_0$ which describes nonscattered photons would be
\be
\bm{I}_0(\tau,x,\mu)=I_{in}(x) e^{\sigma(x)\tau/\mu} (1, 0)^T
\ee 
and the Stokes vector for $k$-th scattering order is obtained from the previous one using \eqref{eq:Rop} and \eqref{eq:Sop}
\be
\bm{I}_k=\hat{\bm{\Sigma}}\hat{\bm{R}}\bm{I}_{k-1}
\ee

Then one can compute the emergent flux $xF_x(x,\mu)$
and the polarization degree of the outgoing radiation
\be
p(x,\mu)=100\%\frac{I(\tau_T,x,\mu)}{Q(\tau_T,x,\mu)}
\ee



\subsection{Compton redistribution matrices} \label{redistr}

The redistribution matrix $\bm{R}(x,\mu,x_1,\mu_1)$
is used in \eqref{eq:Source} and describes the probability for a photon with energy $x_1$ and propagation lean $\mu_1$ to have energy $x$ 
and lean $\mu$ after a scattering.  
Further let all the values that characterize photons before scattering be denoted with subscript 1,
and for photons after scattering have none. 
If one use the logarithmic scale for energies it is more convenient to use the slightly different expression for $\bm{R}$  
$$
\bm{R}^l(y,\mu,y_1,\mu_1)=\frac{x^2}{x_1}\bm{R}(x,\mu,x_1,\mu_1) $$$$
x=e^y \qquad x_1=e^{y_1}
$$
And the equation \eqref{eq:Source} is being written  without the energy factor $x^2/x_1^2$ 
\be
S(\tau,y,\mu)= \int_{-\infty}^\infty dy \int_{-1}^1 d\mu_1 
\bm{R}^l(y,\mu,y_1,\mu_1)\bm{I}(\tau,y_1,\mu_1)
\ee
The the limits of the integral over logarithmic energy $y$ just must be enough 
for the expression below the integral to contain the majority of the photon energies, which are mostly between the characteristic temperatures of electron gas $(\Theta)$ and initial radiation ($T$).

(Nagirner \& Poutanen 1993)

\be
\bm{R}(x,\mu,x_1,\mu_1)=\int_0^{2\pi}d\varphi 
\bm{L}(-\chi)
\bm{R}^t(x,x_1,\xi)
\bm{L}(\chi_1)
\ee


$$
\eta=\sqrt{1-\mu^2} \qquad \eta_1=\sqrt{1-\mu_1^2}
$$
\be
\xi=\cos\theta=\mu \mu_1 + \eta \eta_1 \cos \phi
\ee

\be
\bm{L}(\chi)=
\left( {\begin{array}{cccc}
    1 & 0 & 0 & 0 \\
    0 & \cos{2\chi} & \sin{2\chi} & 0 \\
    0 & \sin{2\chi} & \cos{2\chi} & 0 \\
    0 & 0 & 0 & 1  
   \end{array} } \right)
\ee

\be
\cos\chi=\frac{\mu_1-\mu\xi}{\eta\sin\theta}
\qquad
\cos\chi_1=\frac{\mu-\mu_1\xi}{\eta_1\sin\theta}
\ee

\be
\sin\chi=\frac{\eta_1\sin\varphi}{\sin\theta}
\qquad
\sin\chi_1=-\frac{\eta\sin\varphi}{\sin\theta}
\ee


\be
\cos2\chi=2\frac{(\mu_1-\mu\xi)^2}{\eta^2(1-\xi^2)}-1
\ee
\be
\cos2\chi_1=2\frac{(\mu-\mu_1\xi)^2}{\eta_1^2(1-\xi^2)}-1
\ee
\be
\sin2\chi\sin2\chi_1=-
\frac{(\mu_1-\mu\xi)(\mu-\mu_1\xi)\sin^2\varphi}{(1-\xi^2)^2}
\ee

From the three matrices product only upper left 2 by 2 corner is needed

\be
\bm{R}=
\left( {\begin{array}{cc}
    R_{11} & R_{12}  \\
    R_{21} & R_{22}  
    \end{array} } \right)
\ee
and the components are
$$
R_{11}=\int_0^{2\pi} d\varphi R^t 
$$$$
R_{12}=\int_0^{2\pi} d\varphi R^t_I\cos2\chi_1 
$$$$
R_{21}=\int_0^{2\pi} d\varphi R^t_I\cos2\chi 
$$$$
R_{22}=\int_0^{2\pi} d\varphi (R^t_Q\cos2\chi\cos2\chi_1 + R^t_U \sin2\chi\sin2\chi_1)
$$
The functions $R^t,\,R^t_I,\,R^t_Q,\,R^t_U$ are components of thermal Compton redistribution matrix $\bm{R}^t(x,x_1,\xi)$ 
\be
\bm{R}^t= 
\left( {\begin{array}{cccc}
    R^t & R^t_I & 0 & 0 \\
     R^t_I &  R^t_Q & 0 & 0 \\
    0 & 0 &  R^t_U & 0 \\
    0 & 0 & 0 &  R^t_V  
   \end{array} } \right)
\ee
which is obtained by taking an integral over electron Lorenz factor $\gamma$ with a distribution function $f(\gamma)$ which
in the case is given by the formula \eqref{eq:Maxwell}
\be
\bm{R}^t(x,x_1,\xi)=\frac38\int_{\gamma_*}^\infty d\gamma
f(\gamma)\bm{R}^m(x,x_1,\xi,\gamma)
\ee

$$
q=xx_1(1-\xi)$$$$
Q=\sqrt{(x-x_1)^2-2q}
$$
\be
\gamma_*=\frac{x-x_1+Q\sqrt{1+2/q}}2
\ee

The matrix $\bm{R}^m$ is the Compton redistribution matrix for monoenergetic electron gas. If each electron has a Lorenz factor of $\gamma$ then the redistributon matrix in such gas is
\be
\bm{R}^m(x,x_1,\xi,\gamma)= 
\left( {\begin{array}{cccc}
    R^m & R^m_I & 0 & 0 \\
     R^m_I &  R^m_Q & 0 & 0 \\
    0 & 0 &  R^m_U & 0 \\
    0 & 0 & 0 &  R^m_V  
   \end{array} } \right)
\ee





\end{document}
