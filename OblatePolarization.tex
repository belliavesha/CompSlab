
\documentclass[14pt]{article}
\usepackage[utf8]{inputenc} 
\usepackage[T2A]{fontenc} % before babel!
\usepackage[english]{babel}
\usepackage{amssymb}
\usepackage{epsfig}
\usepackage{color}
\usepackage{bm}
\usepackage{mathtools}
\usepackage{graphicx}
\usepackage{gensymb}
\usepackage{lipsum}
\usepackage{hyperref}   
\usepackage{epigraph}


\newcommand{\be}{\begin{equation}}
\newcommand{\ee}{\end{equation}}
\newcommand*\df {\mathop{}\!\mathrm{d}}


\begin{document}

			The instrument observes the polarized radiation in the picture plane.
			There are several steps one sould make to fetch the the Stokes vektor from the star corotating frame.
			Let us look on the main basis, the basis of the pictre frame, which is formed by $\bm z$ and $\bm k$ 
			\be 
				\label{mbasis}
				\bm{e}_1^m = \frac{\bm{z}-\cos{i} \bm{k}}{\sin{i}},\qquad 
				\bm{e}_2^m = \frac{\bm{k} \times \bm{z}}{\sin{i}}.
			\ee
			The vector  $\bm{e_1^m}$ 
			is collinear to the projection of the star rotation axis ot the picture frame, and the second  vector  $\bm{e_2^m}$ is perpendicular to the first one.
			This basis is fixed in the lab frame, since the pulsar rotation  axis (along the $\bm \omega$) is still and very stableand the line of sight (along the  $\bm{k}$) does not also change within the observation time.
			The polarization vector is observed in this basis.

			Next let us consider the basis formed by $\bm r$ and $\bm k$
			\be\label{rbasis}
			\bm{e}^r_1 = \frac{\bm{r}-\cos{\psi} \bm{k}}{\sin{\psi}},\qquad 
			\bm{e}^r_2 = \frac{\bm{k} \times \bm{r}}{\sin{\psi}}.
			\ee
			This basis describes the same plane as the previous one but it rotates relatively to the main one.
			The difference between the polarization angles in these two bases we denote by $ \chi_0 $.
			This angle is measured from the vector of the main basis $\bm{e}_1^m$ in the counterclockwise direction.
			Using the formula for $\cos\psi$ (\ref{cospsi}) we can obtain the triginimetric functions of this angle 
			\be
			\cos{\chi_0}=\bm{e}_1^m \cdot \bm{e}^r_1 = \frac{\sin{i}\cos{\theta}-\sin{\theta}\cos{i}\cos{\varphi}}{\sin{\psi}} , 
			\ee \be 
			\sin{\chi_0}= - \bm{e}_1^m \cdot \bm{e}^r_2 = \bm{e}_2^m \cdot \bm{e}^r_1 = - \frac{\sin{\theta}\sin{\varphi}}{\sin{\psi}} ,
			\ee
			From these trigonometric functions we get the angle unambiguously. 
			For the spherical and slow rotation that is sufficient, but we have to take into account the relativistic morion and the oblateness of the star shape.

			To take into cosideration the star curviture we look at a yet another basis, which is formed by the radius vector $\bm{r}$ and the direction of the light propagation $\bm{k_0}$ near the star surface
			\be\label{basis}
			\bm{e}_1 = \frac{\bm{r}-\cos{\alpha} \bm{k_0}}{\sin{\alpha}},\qquad 
			\bm{e}_2 = \frac{\bm{k_0} \times \bm{r}}{\sin{\alpha}}=\bm{e}^r_2.
			\ee
			The light trajectories are planar, so the $\bm{e}_2$ and $\bm{e}^r_2$ are equal vectors. The polarization degree and positional angle in this basis are also the same as in the basis denoted by upper index $r$ (\ref{rbasis}). \textbf{(this is seem to be true)}.
			After that we consider the basis associated with the local normal vector  $\bm n $
			\be\label{nbasis}
			\bm{e}_1^n = \frac{\bm{n}-\cos{\sigma} \bm{k_0}}{\sin{\sigma}},\qquad 
			\bm{e}_2^n = \frac{\bm{k_0} \times \bm{n}}{\sin{\sigma}} .
			\ee
			The last two bases are also lie in the same plane. The angle between them we denote by $\chi_1$.
			This angle similarly to the previous one is given by  \be
			\cos{\chi_1}=\bm{e}_1^n \cdot \bm{e}_1 = \frac{\cos\gamma-\cos\alpha\cos\sigma}{\sin{\alpha}\sin\sigma} , 
			\ee $$ 
			\sin{\chi_1}= \bm{e}_2 \cdot \bm{e}^n_1 = \frac{\bm{n} \cdot (\bm{k_0}\times\bm{r} )}{\sin\alpha\sin\sigma}
			= \frac{\bm{k_0} \cdot (\bm{r}\times\bm{n} )}{\sin\alpha\sin\sigma}=
			$$\be
			= \frac{\sin\alpha}{\sin\psi} \frac{ \bm{k} \cdot (\bm{r}\times\bm{n} )}{\sin\alpha\sin\sigma}
			= \frac{ \sin\gamma\sin i \sin\varphi}{\sin\psi\sin\sigma}.
			\ee
			If the star is spherical, on can see that this angle would be exactly $0$.

			The las base is still connected to the fixed lab frame. 
			The last step is to take into account the positional vector rotation due to the relativistic motion of the star surface.
			Let us consider the basis in the spot comoving frame
			\be\label{primebasis}
				\bm{e}_1' = \frac{\bm{n}-\cos{\sigma'} \bm{k_0'}}{\sin{\sigma'}},\qquad 
				\bm{e}_2' = \frac{\bm{k_0'} \times \bm{n}}{\sin{\sigma'}} .
			\ee
			Since \be
				\mu=\cos\sigma'=\delta\cos{\sigma} 
			\ee
			the first vector can be written as 
			\be
			\bm{e}_1' = \frac{\bm{n}+\delta\Gamma\cos{\sigma} (\bm{\beta-k_0})}{\sqrt{1-\mu^2 } }.
			\ee
			The relativistic correction of the polarization vector we denoted by $\chi'$.

			The sine of this angle is
			\be\label{sinchi}
			\sin{\chi'}=\bm{e}_1'\cdot \bm{e}_2^n =
			 \frac{\mu\Gamma\beta }{\sin{\sigma}\sqrt{1-\mu^2} } \bm{\hat\beta} \cdot(\bm{k_0} \times \bm{n}).
			\ee
			where $\bm{\hat\beta}$ is the unity velocity vector.
			
			The scalar triple product $\bm{\hat\beta} \cdot(\bm{k_0} \times \bm{n}) $, in (\ref{sinchi}) is
			\be\label{tripleproductoblate}
			 \bm{k_0} \cdot (\bm{n}\times\bm{\hat\beta} )=\pm
			 \bm{k_0} \cdot \left(\bm{n} \times \frac{\bm{n} \times \bm{r}}{\sin{\gamma}}\right)=\pm
			 \bm{k_0} \cdot \frac{\cos{\gamma}\bm{n} - \bm{r}}{\sin{\gamma}} =\pm
			 \frac{\cos{\sigma}\cos{\gamma}-\cos{\alpha}}{\sin{\gamma}},
			\ee
			Where $+$ sign is for the northern hemisphere and the $-$ is for south.
			Finally, we get
			\be\label{chiprime}
			\sin{\chi'}=\bm{e}_1'\cdot \bm{e}_2^n =\pm
			 \frac{\mu\Gamma\beta (\cos{\sigma}\cos{\gamma}-\cos{\alpha})}{\sin{\gamma}\sin{\sigma}\sqrt{1-\mu^2} }.
			\ee

			That is the simpliest expression, but we only may use it when the $\bm n$ differs from the $\bm r$. 
			Let us then  consider the meridional vector headed towards the north pole
			\be
				\bm m = (- \cos \lambda \cos \phi, -\cos \lambda \sin \phi, \sin \lambda ),
			\ee
			where $\lambda=\theta-\gamma$ is the angle between the normal vector and the spin axis.
			Then we can reweite (\ref{tripleproductoblate}) as
			$$
			 \bm{k_0} \cdot (\bm{n}\times\bm{\hat\beta} )=
			 \bm{k_0} \cdot \bm{m}=\frac1{\sin\psi}(\sin \alpha(\cos i \sin \lambda - \sin i \cos \lambda \cos \phi ) - \sin(\psi-\alpha)\sin\gamma )=	
			$$\be\label{tripleproductpherical}
				(\cos i \sin \lambda - \sin i \cos \lambda \cos \phi+\cos\psi \sin\gamma)\frac{\sin\alpha}{\sin\psi} - \cos \alpha\sin\gamma, 
			\ee
			where we can put small and even zero $\gamma$ angle. And for the small 
			$\psi$
			angles we make use of Beloborodov's approximation.

			The universal formula for the $\sin\chi'$ is then 
			\be
				\sin\chi'={\mu\Gamma\beta }\frac{(\cos i \sin \lambda - \sin i \cos \lambda \cos \phi+\cos\psi \sin\gamma)\frac{\sin\alpha}{\sin\psi} - \cos \alpha\sin\gamma}{\sin{\sigma}\sqrt{1-\mu^2} }.
			\ee
      
			In particular, for the spherical star we have the triple product being 
			\be\label{chiprimespherical}
			\sin{\chi'}=\bm{e}_1'\cdot \bm{e}_2^n =
			 \frac{\mu\Gamma\beta (\cos i \sin \theta - \sin i \cos \theta \cos \phi)}{\sin{\psi}\sqrt{1-\mu^2} }.
			\ee

			The cosine of this angle is obviously always positive, but in case of $\sigma\approx0$ or $ \sigma'\approx0$ it also will be useful  
			$$
				\cos\chi'=\bm{e}_1'\cdot \bm{e}_1^n =\frac{\bm n + \delta  \Gamma \cos\sigma (\bm \beta - \bm{k_0} ) }{\sqrt{1-\mu^2} }\cdot
				\frac{\bm n - \cos \sigma  \bm{k_0} }{\sin{\sigma}} = $$
			$$
				=\frac{1-\cos^2\sigma-\delta \Gamma \cos^2 \sigma \bm \beta \cdot \bm{k_0} }{\sin{\sigma}\sqrt{1-\mu^2} } 
				=\frac{\sin^2\sigma-  \cos^2 \sigma \bm \beta \cdot \bm{k_0} /(1-\bm \beta \cdot \bm{k_0}) }{\sin{\sigma}\sqrt{1-\mu^2} } =
			$$
			$$
				=\frac{\sin^2\sigma -\bm \beta \cdot \bm{k_0} \sin^2\sigma - \cos^2 \sigma \bm \beta \cdot \bm{k_0} }{\sin{\sigma}\sqrt{1-\mu^2} (1-\bm \beta \cdot \bm{k_0}) } 
				=\frac{\sin^2\sigma -\bm \beta \cdot \bm{k_0}  }{\sin{\sigma}\sqrt{1-\mu^2} (1-\bm \beta \cdot \bm{k_0}) } =
			$$\be
				=\frac{\sin^2\sigma + \beta \frac{\sin \alpha }{\sin\psi} \sin i \sin \phi }{\sin{\sigma}\sqrt{1-\mu^2} (1+ \beta \frac{\sin \alpha }{\sin\psi} \sin i \sin \phi) } 
			\ee


			The circular polarization is always zero, so we consider only three componen Stokes vector $(F_I,F_Q,F_U)$.
			In the spot comoving basis the third component is also zero due to the symmetry along the normal.
			The polarization degree then is just $P=F_Q/F_I$.
			The polarization degree is invariant so in the main basis we will have the Stokes vector
			\be
				(F_I, F_Q \cos{2\chi},F_Q \sin{2\chi}),
			\ee
		    where the angle
			\be
			\chi=\chi_0+\chi_1+ \chi'\ee
			is just a sum of the angles between the intermediate bases.


\end{document}
